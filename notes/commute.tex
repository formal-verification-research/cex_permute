\subsubsection*{Automated Threshold Selection for Predicate
	Abstraction for CTMCs}
Use the toggle swith model as an example
(\url{https://github.com/fluentverification/Critical-Values-Prototype/blob/master/ToggleSwitch/control.prism}). 
\begin{enumerate}
	\item Starting with 
	
\end{enumerate}  


Given a CTMC model \model{} and the following transient CSL property
\cslEventually, this method aims at efficiently providing the estimated lower
and upper probability bound, \probMin{} and \probMax, for the path
formula \eventuallyFull, respectively. The proposed method is expected
to work well for highly concurrent model \model{}. 

\subsubsection*{Procedure for generating paths reaching target
	states.}
This procedure can provide a lower probability bound for reaching a
state satisfying \targetSt.

\begin{enumerate}
	\item Generate the first shortest counterexample path from the
	equivalent non-deterministic model \nondetermModel{} of \model{} using PDR. This path starts
	from the initial state \initSt: \ensuremath{\path = \pathFull} and
	the formula \targetSt{} holds in its last state
	\ensuremath{\lastSt=\state{n}}, i.e., \ensuremath{\lastSt \models \targetSt}. 
	\item Construct the set of independent transitions (including empty
	set), \indp{\path}, for path \path\ as \ensuremath{\indp{\path} = \en{\initSt} \cap \en{\state{1}} \cap \dots \cap
		\en{\state{i}} \cap \dots \cap \en{\state{n}}}. Denote
	\en{\state{i}} as the set of enabled transitions in state \state{i}. Essentially, each
	transition \ensuremath{\tran{\alpha} \in \indp{\path}} is enabled in \emph{every} state,
	i.e., \ensuremath{\initSt, \state{1}, \dots, \state{n}} of path
	\path. Therefore, \tran{\alpha} is independent of transitions \tran{0}
	through \tran{n-1}. Also, we need to confirm that executing
	\ensuremath{\tran{\alpha}} from the last state of \path{} reaches
	a target state: \ensuremath{\lastSt=\state{n}
		\xrightarrow{\tran{\alpha}} \state{n+1}} and
	\ensuremath{\state{n+1} \models \targetSt}. 
	\item If \ensuremath{\indp{\path} \neq \emptyset}, consider each transition \ensuremath{\tran{\alpha} \in
		\indp{\path}} and path \ensuremath{\path = \pathFull}. Figure~\ref{fig-interleaveOneTran} illustrates all
	possible path interleavings between \ensuremath{\tran{\alpha}} and
	transitions in \path. There are a total of $(n+1)$ unique paths in this
	figure from \initSt{} to \stAlt. Figure~\ref{fig-interleaveTwoTrans} shows the
	full interleaving of two mutually independent transitions 
	\ensuremath{\tran{\alpha}, \tran{\beta}} and transitions in
	\path. Nodes with the same identifier (e.g., \stAlta{0},
	\stAlta{1}, etc.) represent the same state. There are a total of
	$(n+2)(n+1)$, i.e., $n^2+3n+2$, unique paths in this
	figure from \initSt{} to \stAlta.
	% n=0, 2
	% n=1, 6
	% n=2, 12
	% n=3, 20
	% 1. s0, s1, s2, s3, u3, w3
	% 2. s0, s1, s2, u2, u3, w3
	% 3. s0, s1, s2, u2, w2, w3
	% 4. s0, s1, u1, u2, u3, w3
	% 5. s0, s1, u1, u2, w2, w3
	% 6. s0, s1, u1, w1, w2, w3
	% 7. s0, u0, u1, u2, u3, w3
	% 8. s0, u0, u1, u2, w2, w3
	% 9. s0, u0, u1, w1, w2, w3
	% 10. s0, u0, w0, w1, w2, w3
	% Symmetrically, there are 10 paths to reach w3 through the v-states.
	\begin{itemize}
		\item \textcolor{blue}{How to efficiently enumerate all paths
			from \initSt{} to \stAlt{} illustrated in both figures? To
			calculate the sum of probabilities of all these paths, do we
			have to enumerate all such paths and simulate each one at a
			time? Note that each transition's rate (and hence
			probability) is dependent on its source state. For example,
			executing \ensuremath{\tran{\alpha}} in \initSt{} can have
			different rate (and therefore probability) than executing it
			in \state{n}.}
		\item \textcolor{blue}{How can the method from the previous step be generalized to $k$
			mutually independent transitions in \indp{\path}?} 
	\end{itemize}  
	
	\item Otherwise, i.e., \ensuremath{\indp{\path} = \emptyset}, construct \indp{\path} from the last $k$
	transitions of \path: \ensuremath{\indp{\path} = \en{\state{i}}
		\cap \en{\state{i+1}} \dots \cap \en{\state{i+(k-1)}}} where
	\ensuremath{\state{i+(k-1)} = \lastSt} and
	\ensuremath{\en{\state{i-1}} \cap \en{\state{i}}
		\cap \en{\state{i+1}} \dots \cap \en{\state{i+(k-1)}} =
		\emptyset}. Then permute every transition in \indp{\path} from
	state \state{i} to \state{i+(k-1)}. \textcolor{orange}{Need to add more
		details.}
	
	\item When generating the next shortest counterexample path
	\ensuremath{\path^{\prime}}, choose an outgoing transition \ensuremath{\tran{0}^{\prime}} from
	the initial state \initSt that is \emph{not} in \indp{\path}. If
	\ensuremath{\tran{0}^{\prime}} does not exist, try to find such a
	transition in the next state \state{1}, and so on, until such a
	transition is found.
	
	\item To model check CSL property with upper time bound, consider
	Riley's algorithm to essentially limit the length of generated
	counterexample paths.
	
	\item Repeat the above steps.
\end{enumerate}

To improve efficiency in reconstructing paths, we consider transitions
beyond those in \ensuremath{\indp{\rho}}. \textcolor{OliveGreen}{Determine independence for a finite
	sequence \ensuremath{\tranSeq{i_{1}}{i_{n}}} w.r.t. all
	transitions along a path returned by PDR, even if not every transition in the
	sequence \ensuremath{\tranSeq{i_{1}}{i_{n}}} is independent of all transitions in
	\pathFull, as long as the first transition
	\ensuremath{\tran{i_{1}}} is an independent transition w.r.t. every
	transition in \pathFull.} For example, in the six-reaction model,
transition sequence \ensuremath{\tran{R_{1}}, \tran{R_{2}}} can be executed at any state in the following
shortest path returned by PDR, such that \ensuremath{\state{n} \models
	\targetSt{}}:\\ \ensuremath{\initSt
	\underbrace{\xrightarrow{\tran{R_{4}}} \state{1}
		\xrightarrow{\tran{R_{6}}} \state{2} \xrightarrow{\tran{R_{4}}}
		\state{3} \xrightarrow{\tran{R_{6}}} \state{4} \dots}_{\text{\tran{R_{4}}
			and \tran{R_{6}} repeatedly execute 9 times}} \xrightarrow{\tran{R_{4}}} \state{n}}. Transition
\ensuremath{\tran{R_{1}}} is enabled in every state along this path,
but \ensuremath{\tran{R_{2}}} is not. However, executing
\ensuremath{\tran{R_{1}}} enables \ensuremath{\tran{R_{2}}} and then
executing \ensuremath{\tran{R_{2}}} \emph{does not} alter the
enabledness of \ensuremath{\tran{R_{4}}} or \ensuremath{\tran{R_{6}}}
in every state along this path. Similarly sequence
\ensuremath{\tran{R_{1}}, \tran{R_{3}}} is also such a
sequence. \textcolor{orange}{Consider building a dependency graph for
	syntactic transitions in the PRISM model, as defined in Def. 4.2 of Valmari2011\_CanStubbornSetsBeOptimal\_journal.pdf.}

\subsubsection*{Procedure for generating paths reaching non-target
	states. }
This procedure attempts to provide an \emph{upper} probability bound for reaching the
target state set. Since the CSL property \cslEventually\ is of interest,
then this procedure finds and reconstructs paths that reach states
satisfying \ensuremath{\lnot\targetSt}. Using the steps above, we can
find the lower probability bound for
\ensuremath{\prob(\eventuallyNot)}, \ensuremath{\prob_{min}
	(\eventuallyNot)}. From the principle of duality, we know that
\ensuremath{\eventually = \lnot \alwaysNot}. If one execution (i.e., a
path) satisfies \eventually, then it does not satisfy \alwaysNot, and
vice versa. Therefore, \ensuremath{\prob(\eventually) = 1 -
	\prob(\alwaysNot)}. Therefore, \ensuremath{\probMax{\eventually} =
	1- \probMin{\alwaysNot}}. We can use the procedure for generating
paths reaching target states described above with the target state set
satisfying \alwaysNot. Then we can potentially calculate
\ensuremath{\probMin{\alwaysNot}} to obtain \ensuremath{\probMax{\eventually}}.

Our assumption is that \eventuallyFull\ is a rare event, so evaluating
\cslEventually\ directly in STAMINA may incur challenges. However,
since \ensuremath{\prob(\eventually) = 1 - \prob(\alwaysNot)}, we can
use STAMINA to obtain a probability window for
\ensuremath{\cslAlwaysNot}: \ensuremath{[ \probMin{\alwaysNot},
	\probMax{\alwaysNot} ]}. 

\begin{figure}[tbhp]
	\begin{tikzpicture}
		%%% state-transition from s0 to sn
		\node (s0_h0) {\ensuremath{\state{} = \initSt}};
		\node (t0_h0) [right of=s0_h0, label=above:\tran{0}] {};
		\node (s1_h0) [right of=t0_h0] {\state{1}};
		\node (t1_h0) [right of=s1_h0, label=above: \tran{1}] {};
		\node (s2_h0) [right of=t1_h0] {\state{2}};
		\node (t2_h0) [right of=s2_h0, label=above: \tran{2}] {};
		\node (ddd1) [right of=t2_h0] {$\cdots$}; % For the dots, do the same
		\node (tnm2_h0) [right of=ddd1, label=above:\tran{n-2}] {};
		\node (snm1_h0) [right of=tnm2_h0] {\state{n-1}};
		\node (tn_h0) [right of=snm1_h0, label=above:\tran{n-1}] {};  
		\node (sn_h0) [right of=tn_h0] {\state{n}};
		% Draw horizontal arrows
		\draw [arrow] (s0_h0) -- (s1_h0);
		\draw [arrow] (s1_h0) -- (s2_h0);
		\draw [arrow] (s2_h0) -- (ddd1);		
		\draw [arrow] (ddd1) -- (snm1_h0);
		\draw [arrow] (snm1_h0) -- (sn_h0);
		
		%%% state-transition one row below s0 ... sn
		% Use the below of, left of, right of, etc... to place nodes. Or you can use coordinates                
		\node (t0_v0) [below of=s0_h0, label=left:\ensuremath{\tran{\alpha}}] {};
		\node (s0_hm1) [below of=t0_v0] {\stAlt{0}};
		\node (t1_v0) [below of=s1_h0, label=left:\ensuremath{\tran{\alpha}}] {};
		\node (s1_hm1) [below of=t1_v0] {\stAlt{1}};
		\node (t2_v0) [below of=s2_h0, label=left:\ensuremath{\tran{\alpha}}] {};
		\node (s2_hm1) [below of=t2_v0] {\stAlt{2}};
		% Blank node used to create proper spacing between the dots
		\node (blank) [below of=ddd1] {};
		\node (ddd2) [below of=blank] {$\cdots$};
		\node (t3_v0) [below of=snm1_h0, label=left:\ensuremath{\tran{\alpha}}] {};
		\node (snm1_hm1) [below of=t3_v0] {\stAlt{n-1}};
		\node (t4_v0) [below of=sn_h0, label=left:\ensuremath{\tran{\alpha}}] {};
		\node (sn_hm1) [below of=t4_v0] {\ensuremath{\stAlt{n} = \stAlt{}}};
		% Draw vertical arrows
		% Syntax for arrow is \draw [arrow-style (in our case ``arrow'')] (from) -- (to)
		\draw [arrow] (s0_h0) -- (s0_hm1);
		\draw [arrow] (s1_h0) -- (s1_hm1);
		\draw [arrow] (s2_h0) -- (s2_hm1);
		\draw [arrow] (snm1_h0) -- (snm1_hm1);
		\draw [arrow] (sn_h0) -- (sn_hm1);
		% Draw horizontal arrows
		\node (t0_hn1) [right of=s0_hm1, label=above:\tran{0}] {};
		\draw [arrow] (s0_hm1) -- (s1_hm1);
		\node (t1_hn1) [right of=s1_hm1, label=above:\tran{1}] {};
		\draw [arrow] (s1_hm1) -- (s2_hm1);
		\node (t2_hn1) [right of=s2_hm1, label=above:\tran{2}] {};
		\draw [arrow] (s2_hm1) -- (ddd2);
		\node (tnm2_hn1) [right of=ddd2, label=above:\tran{n-2}] {};
		\draw [arrow] (ddd2) -- (snm1_hm1);
		\node (tn_hn1) [right of=snm1_hm1, label=above:\tran{n-1}] {};
		\draw [arrow] (snm1_hm1) -- (sn_hm1);
	\end{tikzpicture}
	\caption{Interleaving of \ensuremath{\tran{\alpha}} and transitions in \path, where \ensuremath{\tran{\alpha} \in
			\indp{\path}} and \ensuremath{\path = \pathFull}.}
	\label{fig-interleaveOneTran}
\end{figure}

\begin{figure}[tbhp]
	\begin{tikzpicture}[node distance = 1cm] % Adjust the distance here for more or less spacing
		% Each corner is a node
		% Even the \tran{\alpha} and \tran should be nodes
		% HOWEVER, we will use the label=position:value to place them not in
		% the lines.
		%%% state-transition from s0 to sn
		\node (s0_h0) {\ensuremath{\state{} = \initSt}};
		\node (t0_h0) [right of=s0_h0, label=above:\tran{0}] {};
		\node (s1_h0) [right of=t0_h0] {\state{1}};
		\node (t1_h0) [right of=s1_h0, label=above: \tran{1}] {};
		\node (s2_h0) [right of=t1_h0] {\state{2}};
		\node (t2_h0) [right of=s2_h0, label=above: \tran{2}] {};
		\node (ddd1) [right of=t2_h0] {$\cdots$}; % For the dots, do the same
		\node (tnm2_h0) [right of=ddd1, label=above:\tran{n-2}] {};
		\node (snm1_h0) [right of=tnm2_h0] {\state{n-1}};
		\node (tn_h0) [right of=snm1_h0, label=above:\tran{n-1}] {};  
		\node (sn_h0) [right of=tn_h0] {\state{n}};
		% Draw horizontal arrows
		\draw [arrow] (s0_h0) -- (s1_h0);
		\draw [arrow] (s1_h0) -- (s2_h0);
		\draw [arrow] (s2_h0) -- (ddd1);		
		\draw [arrow] (ddd1) -- (snm1_h0);
		\draw [arrow] (snm1_h0) -- (sn_h0);
		
		%%% state-transition one row below s0 ... sn
		% Use the below of, left of, right of, etc... to place nodes. Or you can use coordinates                
		\node (t0_v0) [below of=s0_h0, label=left:\ensuremath{\tran{\alpha}}] {};
		\node (s0_hm1) [below of=t0_v0] {\stAlt{0}};
		\node (t1_v0) [below of=s1_h0, label=left:\ensuremath{\tran{\alpha}}] {};
		\node (s1_hm1) [below of=t1_v0] {\stAlt{1}};
		\node (t2_v0) [below of=s2_h0, label=left:\ensuremath{\tran{\alpha}}] {};
		\node (s2_hm1) [below of=t2_v0] {\stAlt{2}};
		% Blank node used to create proper spacing between the dots
		\node (blank) [below of=ddd1] {};
		\node (ddd2) [below of=blank] {$\cdots$};
		\node (t3_v0) [below of=snm1_h0, label=left:\ensuremath{\tran{\alpha}}] {};
		\node (snm1_hm1) [below of=t3_v0] {\stAlt{n-1}};
		\node (t4_v0) [below of=sn_h0, label=left:\ensuremath{\tran{\alpha}}] {};
		\node (sn_hm1) [below of=t4_v0] {\ensuremath{\stAlt{n}}};
		% Draw vertical arrows
		% Syntax for arrow is \draw [arrow-style (in our case ``arrow'')] (from) -- (to)
		\draw [arrow] (s0_h0) -- (s0_hm1);
		\draw [arrow] (s1_h0) -- (s1_hm1);
		\draw [arrow] (s2_h0) -- (s2_hm1);
		\draw [arrow] (snm1_h0) -- (snm1_hm1);
		\draw [arrow] (sn_h0) -- (sn_hm1);
		% Draw horizontal arrows
		\node (t0_hn1) [right of=s0_hm1, label=above:\tran{0}] {};
		\draw [arrow] (s0_hm1) -- (s1_hm1);
		\node (t1_hn1) [right of=s1_hm1, label=above:\tran{1}] {};
		\draw [arrow] (s1_hm1) -- (s2_hm1);
		\node (t2_hn1) [right of=s2_hm1, label=above:\tran{2}] {};
		\draw [arrow] (s2_hm1) -- (ddd2);
		\node (tnm2_hn1) [right of=ddd2, label=above:\tran{n-2}] {};
		\draw [arrow] (ddd2) -- (snm1_hm1);
		\node (tn_hn1) [right of=snm1_hm1, label=above:\tran{n-1}] {};
		\draw [arrow] (snm1_hm1) -- (sn_hm1);
		
		%%% state-transition two rows below s0 ... sn
		% Use the below of, left of, right of, etc... to place nodes. Or you can use coordinates                
		\node (t0_v1) [below of=s0_hm1, label=left:\ensuremath{\tran{\beta}}] {};
		\node (s0_hm2) [below of=t0_v1] {\stAlta{0}};
		\node (t1_v1) [below of=s1_hm1, label=left:\ensuremath{\tran{\beta}}] {};
		\node (s1_hm2) [below of=t1_v1] {\stAlta{1}};
		\node (t2_v1) [below of=s2_hm1, label=left:\ensuremath{\tran{\beta}}] {};
		\node (s2_hm2) [below of=t2_v1] {\stAlta{2}};
		% Blank node used to create proper spacing between the dots
		\node (blank2) [below of=ddd2] {};
		\node (ddd3) [below of=blank2] {$\cdots$};
		\node (t3_v1) [below of=snm1_hm1, label=left:\ensuremath{\tran{\beta}}] {};
		\node (snm1_hm2) [below of=t3_v1] {\stAlta{n-1}};
		\node (t4_v1) [below of=sn_hm1, label=left:\ensuremath{\tran{\beta}}] {};
		\node (sn_hm2) [below of=t4_v1] {\ensuremath{\stAlta{n} = \stAlta{}}};
		% Draw vertical arrows
		% Syntax for arrow is \draw [arrow-style (in our case ``arrow'')] (from) -- (to)
		\draw [arrow] (s0_hm1) -- (s0_hm2);
		\draw [arrow] (s1_hm1) -- (s1_hm2);
		\draw [arrow] (s2_hm1) -- (s2_hm2);
		\draw [arrow] (snm1_hm1) -- (snm1_hm2);
		\draw [arrow] (sn_hm1) -- (sn_hm2);
		% Draw horizontal arrows
		\node (t0_hn1) [right of=s0_hm2, label=above:\tran{0}] {};
		\draw [arrow] (s0_hm2) -- (s1_hm2);
		\node (t1_hn1) [right of=s1_hm2, label=above:\tran{1}] {};
		\draw [arrow] (s1_hm2) -- (s2_hm2);
		\node (t2_hn1) [right of=s2_hm2, label=above:\tran{2}] {};
		\draw [arrow] (s2_hm2) -- (ddd3);
		\node (tnm2_hn1) [right of=ddd3, label=above:\tran{n-2}] {};
		\draw [arrow] (ddd3) -- (snm1_hm2);
		\node (tn_hn1) [right of=snm1_hm2, label=above:\tran{n-1}] {};
		\draw [arrow] (snm1_hm2) -- (sn_hm2);
		
		%%% state-transition one row ABOVE s0 ... sn
		% Use the below of, left of, right of, etc... to place nodes. Or you can use coordinates                
		\node (t0_v2) [above of=s0_h0, label=left:\ensuremath{\tran{\beta}}] {};
		\node (s0_hp1) [above of=t0_v2] {\stAltb{0}};
		\node (t1_v2) [above of=s1_h0, label=left:\ensuremath{\tran{\beta}}] {};
		\node (s1_hp1) [above of=t1_v2] {\stAltb{1}};
		\node (t2_v2) [above of=s2_h0, label=left:\ensuremath{\tran{\beta}}] {};
		\node (s2_hp1) [above of=t2_v2] {\stAltb{2}};
		% Blank node used to create proper spacing between the dots
		\node (blank3) [above of=ddd1] {};
		\node (ddd3) [above of=blank3] {$\cdots$};
		\node (t3_v2) [above of=snm1_h0, label=left:\ensuremath{\tran{\beta}}] {};
		\node (snm1_hp1) [above of=t3_v2] {\stAltb{n-1}};
		\node (t4_v2) [above of=sn_h0, label=left:\ensuremath{\tran{\beta}}] {};
		\node (sn_hp1) [above of=t4_v2] {\ensuremath{\stAltb{n}}};
		% Draw vertical arrows
		% Syntax for arrow is \draw [arrow-style (in our case ``arrow'')] (from) -- (to)
		\draw [arrow] (s0_h0) -- (s0_hp1);
		\draw [arrow] (s1_h0) -- (s1_hp1);
		\draw [arrow] (s2_h0) -- (s2_hp1);
		\draw [arrow] (snm1_h0) -- (snm1_hp1);
		\draw [arrow] (sn_h0) -- (sn_hp1);
		% Draw horizontal arrows
		\node (t0_hp1) [right of=s0_hp1, label=above:\tran{0}] {};
		\draw [arrow] (s0_hp1) -- (s1_hp1);
		\node (t1_hp1) [right of=s1_hp1, label=above:\tran{1}] {};
		\draw [arrow] (s1_hp1) -- (s2_hp1);
		\node (t2_hp1) [right of=s2_hp1, label=above:\tran{2}] {};
		\draw [arrow] (s2_hp1) -- (ddd3);
		\node (tnm2_hp1) [right of=ddd3, label=above:\tran{n-2}] {};
		\draw [arrow] (ddd3) -- (snm1_hp1);
		\node (tn_hp1) [right of=snm1_hm1, label=above:\tran{n-1}] {};
		\draw [arrow] (snm1_hp1) -- (sn_hp1);
		
		%%% state-transition two rows ABOVE s0 ... sn
		% Use the below of, left of, right of, etc... to place nodes. Or you can use coordinates                
		\node (t0_v3) [above of=s0_hp1, label=left:\ensuremath{\tran{\alpha}}] {};
		\node (s0_hp2) [above of=t0_v3] {\stAlta{0}};
		\node (t1_v3) [above of=s1_hp1, label=left:\ensuremath{\tran{\alpha}}] {};
		\node (s1_hp2) [above of=t1_v3] {\stAlta{1}};
		\node (t2_v3) [above of=s2_hp1, label=left:\ensuremath{\tran{\alpha}}] {};
		\node (s2_hp2) [above of=t2_v3] {\stAlta{2}};
		% Blank node used to create proper spacing between the dots
		\node (blank4) [above of=ddd3] {};
		\node (ddd4) [above of=blank4] {$\cdots$};
		\node (t3_v3) [above of=snm1_hp1, label=left:\ensuremath{\tran{\alpha}}] {};
		\node (snm1_hp2) [above of=t3_v3] {\stAlta{n-1}};
		\node (t4_v3) [above of=sn_hp1, label=left:\ensuremath{\tran{\alpha}}] {};
		\node (sn_hp2) [above of=t4_v3] {\ensuremath{\stAlta{n} = \stAlta{}}};
		% Draw vertical arrows
		% Syntax for arrow is \draw [arrow-style (in our case ``arrow'')] (from) -- (to)
		\draw [arrow] (s0_hp1) -- (s0_hp2);
		\draw [arrow] (s1_hp1) -- (s1_hp2);
		\draw [arrow] (s2_hp1) -- (s2_hp2);
		\draw [arrow] (snm1_hp1) -- (snm1_hp2);
		\draw [arrow] (sn_hp1) -- (sn_hp2);
		% Draw horizontal arrows
		\node (t0_hp1) [right of=s0_hp2, label=above:\tran{0}] {};
		\draw [arrow] (s0_hp2) -- (s1_hp2);
		\node (t1_hp1) [right of=s1_hp2, label=above:\tran{1}] {};
		\draw [arrow] (s1_hp2) -- (s2_hp2);
		\node (t2_hp1) [right of=s2_hp2, label=above:\tran{2}] {};
		\draw [arrow] (s2_hp2) -- (ddd4);
		\node (tnm2_hp1) [right of=ddd4, label=above:\tran{n-2}] {};
		\draw [arrow] (ddd4) -- (snm1_hp2);
		\node (tn_hp1) [right of=snm1_hm1, label=above:\tran{n-1}] {};
		\draw [arrow] (snm1_hp2) -- (sn_hp2);
	\end{tikzpicture}
	\caption{Interleaving of \ensuremath{\tran{\alpha}, \tran{\beta},} and
		transitions in \path, where \ensuremath{\tran{\alpha}, \tran{\beta} \in
			\indp{\path}} and \ensuremath{\path = \pathFull}. Note that \tran{\alpha} and
		\tran{\beta} are also assumed to be independent of each other.}
	\label{fig-interleaveTwoTrans}
\end{figure}

\subsubsection*{Optimize PDR's counterexample generation with
	permutation of independent transitions.} For the 300 traces PDR
generated for the six-reaction model, many of them can be produced by
permuting transition r\_one at different locations of the first (i.e.,
the shortest) path returned by PDR. 


\subsubsection*{Can independent transition relation help inductive
	invariant generation for PDR and/or PrIC3?} Need to look into
whether and how an independent transition informs about one step
relative inductiveness.

\subsubsection*{Assume-guarantee view of chemical reaction network?}
Use the six-reaction model as an example. Based on the initial
condition and the property of interest \targetSt, we first identify
transition(s) that can help to reach a target state satisfying
\targetSt. 

If we model each reaction as a separate object in IVy, can we use
assume-guarantee reasoning to check for reachability and generate paths?

Utilize mutual induction (like the ping-pong example) between R4 and
its environment, i.e., composition of other five objects, up to $k$
steps?