\subsection{Using Single Enabled Transitions}

Consider a trace \ensuremath{\pathFull}. At any state \state{k}, it is likely that several transitions are enabled. Using that knowledge, we propose the following process to find a very new path after commuting once:

\begin{enumerate}
	\item Starting with state \state{k}, take one of the available transitions. This transition should not be the independent transition or the transition in the existing path. Then, you will be in \state{k+1}'.
	\item Set \state{k+1}' as the new initial state in an IVy model. Use PDR to generate a path from \state{k+1}' to \state{n} (such that \state{n} $\models \targetSt$, \state{n} is a target state)
	\item Use the newly generated path as the seed path. Starting with either $k=0$ or $k=n$, generate and commute more paths.
\end{enumerate}


\subsection{Disabling a Transition}

Consider the following process:

\begin{enumerate}
	\item In the IVy model, disable a single transition \tran{k} in some way. Keep the rest of the model the same, including the initial state and property to check. Ways to disable \tran{k} could include:
	\begin{itemize}
		\item Using an invariant and flag variable, never allow the transition to have been fired (probably the slowest option)
		\item Comment out the transition entirely (probably the fastest option)
		\item Permanently set the transition's guard to false
	\end{itemize}
	\item Check the IVy model. Either result can give useful information to a user:
	\begin{itemize}
		\item If we get a trace back, we know that \tran{k} is not required to reach the target from the initial state. We also have a fresh (and very different) seed path to try our commuting algorithm on.
		\item If we get \textit{unreachable} back, the model will not ever reach the target state without using \tran{k}. We do not get a new path, but we can inform the user that the model will never reach the target without using \tran{k}, so modifying the probability or existence of \tran{k} can help modify the probability of reaching the target
	\end{itemize}
	\item Repeat for all $\tran{k} \in \transet$.
\end{enumerate}